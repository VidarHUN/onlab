%                                             -*- coding: utf-8 -*-
% Mindenkinek csak javasolni tudjuk, hogy latex-et használjon.
% Szakdolgozatnál vagy diplománál már egyértelműen kijönnek az
% előnyei a Worddel szemben.  Ennek ellenére ez a sablon messze nem
% tökéletes.  Ha valamit javítanál benne, kérlek, küld vissza, hogy
% hallgatótársaid is profitáljanak belőle.  Köszönöm.

% További nehézséget okoz, hogy a népszerű latex disztribúciók nem
% tartalmazzák a legújabb változatát a magyar.ldf-nek.  A szükséges
% fájlokat a sablon mellé bemásoltuk, de le is tölthetőek innen:
% http://www.math.bme.hu/latex/
%
%
%
\documentclass[a4paper,oneside]{article}
\usepackage[margin=3cm]{geometry}
% =================================================================
% Magyar nyelvi támogatás
%------------------------
% ###################
% Nyelvváltó parancsok:
%\selectlanguage{english}
%\selectlanguage{magyar}
% rövid angol beszúrás:  \foreignlanguage{english}{some english text}
% határozott névelők generálása ``magyar'' babel-el:
% argumentum+megfelelő határozott nevelő: \az{},\Az{}
% csak a megfelelő határozott nevelő: \az*{}, \Az*{}
% címkék: \aref{}, \aref*{}, képletekhez \aref()
%        \Aref{}, \Aref*{}, képletekhez \Aref()
% oldalak: \apageref{}, \apageref*{}
%        \Apageref{}, \Apageref*{}
% idézetek: \acite, \acite*, \Acite, \Acite*
% ###################
\usepackage[english,magyar]{babel} %vegyes nyelvi támogatás a
% magyar helyesírás ellenőrzéshez (ispell) és elválasztáshoz
\selectlanguage{magyar}

%=================================================================
% direkt ékezetes karakter beírás támogatás
%-------------------------------------------
\usepackage[T1]{fontenc}
\usepackage[utf8]{inputenc}
\usepackage{multirow} 
%================================================================
% Undorító dolog bitmappelt (Type III) betűtípust nézni a PDF-ben
% képernyőn. Az alapértelmezett Computer Modern font LaTex-ben
% bitmappelt, ezért használjunk Times fontot:
\usepackage{times}

%================================================================
% ha ábrát akarunk beemelni, akkor használjuk a graphicx/graphics
% csomagot és az \includegraphics[width=<width>]{abra.pdf} parancsot
\usepackage{graphicx} %for graphics
%kepek helye a gyokerhez(ehhez a file-hoz kepest) kepest
\graphicspath{{./figs/}}

%================================================================
% Kötelezően használjuk a hyperref csomagot, mert ezzel többek között 
%  kultúrált hyperlinkelt PDF-et lehet csinálni az alábbi
%  variációkban, különféle hyperref backend-ekkel:
%  pdflatex,dvipdfm,ps2pdf
% tapsztalataim szerint a MikTeX (Win32) a 'dvipdfm' konverzióval
% optimális  míg a teTeX (Linux/Solaris) jobb szereti a 'dvips' módszert
%------------------------------------
% pontosan egyet kommentezzünk be!!!!!!!
% értelemszerűen backend függően generáljunk dvi-ból PDF-et!!!
%------------------------------------
% A hyperref csomag az utolsó beolvasott csomag legyen, kivéve néhány
% problémás csomagot, pl. algorithm
%-----------
% ########################### FONTOS ###########################
% A hyperref hibásan működik a babel csomag 'magyar.ldf' fájljának
% 1.5-ös verziójánál korábbi változatával. 2004. februárjában a MikTeX
% és teTex disztribúciók még csak a v.1.4 verziót tartalmazták! A fájl
% aktuális verziója a BME Matematikai intézet LaTeX honlapjáról
% elérhető: http://www.math.bme.hu/latex/ 
% A lusták kedvéért a jelen sablon mellé is mellékelem:
% magyarlatex_0.01-2.tar.gz 
% ########################### FONTOS ###########################
%-----------
\usepackage[colorlinks=true]{hyperref}

%%%%%%%%%%%%%%%%%%%%%%%%%%%%%%%%%%%%%%%%%%%%%%%%%%%%%%%%%%%%%%%%%%%
% Itt kezdődik maga a dokumentum
%%%%%%%%%%%%%%%%%%%%%%%%%%%%%%%%%%%%%%%%%%%%%%%%%%%%%%%%%%%%%%%%%%
\begin{document}
\input{onlabmacros} % Ez kell!!!
\markright{Váradi Richárd Tamás (XA5OZH)} % egyoldalas fejléc!!!
%--------------------------------------------------------------------
% fedlap
%--------------------------------------------------------------------
\begin{titlepage}
%bme logo 
 \begin{figure}[h]
    \centering
      \includegraphics[width=12cm]{bme_logo}
  \label{fig:bme_logo}
  \end{figure}
  \thispagestyle{empty}
  %cím generálás
  \onlabcim

% \begin{center}
%   \begin{tabular}{ p{3cm} p{5cm} }
%   
%   Készítette: & Beszámoló Péter  \\
%   Neptun-kód: & BPOX43  \\
%   Ágazat: & Médiainformatika  \\
%   E-mail cím: & b.peter@onlab.hu  \\
%   Konzulens: & Dr. Péhádes István  \\
%   E-mail cím: & pehades@tmit.bme.hu  \\
%   Konzulens: & Doktor Andusz  \\
%   E-mail cím: & doktora@tmit.bme.hu  \\
%   
%   \end{tabular}
% \end{center}

 
  %\szerzo argumentumok: #1=Név, #2=Neptunkód, #3=szakirány, #4=email,#5 konzulens-1, #6 konzulens-1-email, #7 konzulens-2, #8 konzulens-2-email
  \onlabszerzo{Váradi Richárd Tamás}{XA5OZH}{Médiainformatika}{ricsi19981007@gmail.com}{Dr. Rétvári Gábor}{retvari@tmit.bme.hu}{}{}
 
 
%\feladatcim argumentuma a feladat rövid, 1 soros címe
  \feladatcim{A Kubernetes és a szolgáltatáshálók hálózati kérdései} 

  %\feladatmaga argumentuma a feladat 1-2 bekezdésnyi ismertetése
  \feladatmaga{A Kubernetes és szolgáltatáshálók hálózati megoldásainak feltérképezése volt. 
  Az egyik olyan probléma, amit én és a konzulensem találtunk, hogy az Istio, mint szolgáltatásháló 
  nem képes UDP (User Datagram Protocol) csomagok fogadására. Viszont az Istio-ban használt Envoy proxy 
  egy újabb verziója már képes UDP csomagokat továbbítani így megvalósítható az is, hogy az Istio-ba 
  ilyen forgalmat irányítsunk be. A félév során egy olyan megoldást kellett találnom, amivel az Istio elé 
  ezzel a proxy-val képes legyek egy úgynevezett átjárót létrehozni. Ennek az átjárónak UDP csomagokat kell 
  tudnia fogadni és átalakítani őket olyan formátumúvá, amit az Istio képes lekezelni. }

 
  %\tanevfelev argumentumok:
  % #1=Tanév (xxxx/xx alakban), #2=félév (pont nélkül!)
  
  \tanevfelev{2019/20}{II}
 
\end{titlepage} 

%==================================================================
\section{A laboratóriumi munka környezetének ismertetése,
     a munka előzményei és kiindulási állapota}
\label{sec:kornyezet}
% A munka  előzményei és kiindulási állapota
% \newpage
\subsection{Bevezető}
\label{sec:bevezeto}
Az ötlet alapvetően egy GitHub-os hibajegyből született. A GitHub egy 
webalapú verziókezelő és együttműködésű felület szoftverfejlesztőknek. 
A hibajegyek úgy születnek, hogy projektekhez lehet őket hozzáírni és erre jó 
eséllyel a fejlesztők vagy más felhasználók reagálnak. Számunkra most ez a jegy 
lesz a fontos: \url{https://github.com/istio/istio/issues/1430}. Pár példát szeretnék 
idézni és fordítani, hogy miért kellene ez a támogatottság: 
\begin{itemize}
	\item  Publikus felhők, ahol egy titkosított szolgáltatáshálót hoznának létre, amely 
	képes kezelni olyan dolgokat, mint a DNS (Domain Name System) és NTP (Network Time Protocol).
	\item IoT (Internet of Things) területen.
	\item Telekommunikáció terén is jó lehet a 3G-től használják az UDP protokollt adattovábbításra. 
\end{itemize}
Ennek megvalósítására a konzulensemmel úgy gondoltuk, hogy érdemes lenne 
egy átjárót felépíteni az Istio elé, amelyet az alábbi ábrákon be is mutatok. 
\begin{center}
\includegraphics[width=\textwidth]{teljes_architektura}
\end{center}
Az \textbf{a} lehetőség nem teljesen biztos, hogy megvalósítható az Envoy korlátai 
miatt. Az Envoy egy nyílt forráskódú perem és szolgáltatás proxy, ami natív felhőalkalmazásokhoz 
lett írva C++-ban. Ez a megvalósítás elméleti szinten működőképes lenne, 
viszont nagyon kevés dokumentáció található arról, hogy az UDS (Unix Domain Socket) 
pontosan hogyan működik az Envoy-ban. \\
A \textbf{b} megoldás annyiban egyszerűbb, hogy ott kettő konténert hozunk létre egy pod-ban, ahol 
az Envoy UDP csomagokat figyel és UDP csomagokat továbbít egy Alpine alapú konténernek, amiben fut egy 
socket parancs, ami képes UDP forgalmat UDS-re irányítani. A Pod a kubernetes legkisebb egysége, 
amelyekben konténereket lehet létrehozni. A konténereket érdemes úgy létrehozni, hogy 
konténerenként egy folyamat fusson benne. Az Alpine egy nagyon kicsi erőforrás igényű unix alapú 
operációs rendszer. 
%<Mit kell tudni a feladatról, esetleges elméleti bevezető (nagyon
%értelemszerű dolgokat ne definiáljunk, de jó, ha egy kicsit
%kontextusba kerül a témakör, miért fontos ez nekünk, mi volt eddig,
%milyen megoldások jöhetnek szóba és miért emellett döntöttünk, milyen
%kari nagyobb projektbe kapcsolódik ez), stb. Terjedelem max. 50\%
%beszámolónak.>

%Ennek a résznek az a szerepe, hogy az olvasó számára megmutassa az
%elvégzett munka tágabb környezetét. Ez a rész lehet megegyező tartalmú
%más, ugyanazon a témán dolgozó hallgatókéval, de akkor mindenképpen
%tüntessük fel, hogy kivel dolgoztatok együtt, és hogy pontosan, hogy
%osztottátok meg a munkát.

\subsection{Elméleti összefoglaló}
\subsubsection{Docker}
A Docker egy szolgáltatáskészlet Paas (Platform as a Service) termékcsalád, 
amelyek operációs rendszer szintű virtualizációt végeznek, hogy a szoftvert 
csomagokban, úgynevezett konténerekben lehessen létrehozni. 

A konténerek egymástól elválasztva és saját szoftverüket futtatva léteznek 
külön könyvtárakkal és konfigurációs fájlokkal. Ezek a konténerek képesek 
kommunikálni egymással jól definiált csatornákon keresztül. Érdemes azt az 
elvet szem előtt tartani, hogy minden konténerben csak egyetlen folyamat fusson 
így sokkal egyszerűbb esetleges hibánál a forrást megtalálni és javítani. 

Minden konténer egyetlen operációs rendszer kernelt használ, ezért sokkal 
kevesebb erőforrást igényelnek, mint a rendes virtuális gépek. Viszont 
olyan hátulütője van ennek a tulajdonságának, hogy mondjuk a kiszolgáló 
egy linux operációs rendszer, akkor csak linux alapú konténerek hozhatóak 
létre, míg egy windows alapú kiszolgálónál csak windows alapú konténerek 
születhetnek. Bár az utóbbi állítás a Hyper-V újítása révén már nem probléma 
érdekességként ajánlom a Microsoft dokumentációját erről ~\cite{linuxwindows}

Ezek a konténereket képfájlokból lehet kialakítani, amiket saját magunk is 
megírhatunk, de böngészhetünk is közülük a DockerHub-on ~\cite{dockerhub}, 
amely hasonlóan működik, mint a GitHub. 

Mikor egy ilyen képfájlt létrehozunk, akkor mindig kell egy alap képfájl, 
amelyet az említett DockerHub oldalról letudunk tölteni a docker alkalmazás 
segítségével. 

További információ gyűjtésére erről a témáról a következő oldalakat ajánlom:
~\cite{dockeroff} ~\cite{dockerwiki}

\subsubsection{Kubernetes}
A kubernetes egy nyílt forráskódú rendszer, amely a fejlesztés automatizálására, 
skálázásához és konténer alapú alkalmazások kezelésére való. Eredetileg a 
Google fejlesztette, de jelenleg a CNCF (Cloud Native Computing Foundation) 
tartja karban.

Azért érdemes használni, mert mikroszolgáltatások lehet benne létrehozni, amelyek 
tudnak egymással kommunikálni, de a hálózati megvalósítások nem ezekbe a szolgáltatásokban 
kell létrehozni, mert a kubernetes erről gondoskodik nekünk olyan hálózati technológiákkal, 
amelyek megtalálhatóak egy átlagos hálózatban is. Ilyen például a DNS, Routing táblák,
IP (Internet Protocol) táblák.

A következőkben ismertetem a két legalapvetőbb komponenst, amelyek a Pod és a Node. 

A \textbf{Pod} a legkisebb létrehozható objektum alkalmazás fejlesztésére. Egyetlen 
Pod egy futó folyamatot reprezentál a klaszterünkben és egy vagy több Docker 
konténert tartalmaz, amelyek saját tárhelyez igényel és egyedi IP címet. 
Ezek a konténerek úgy lettek tervezve, hogy ugyanazon a gépen helyezkedjenek el 
és legyenek egyszerre ütemezve. 

A klaszter úgynevezett dolgozó gépek gyűjteménye, amelyeket Node-oknak nevezünk. 
Minden klaszternek legalább egy dolgozó node-ja van. 

Mint már említettem egy node egy klaszterben dolgozó gépet reprezentál, ezek 
lehetnek fizikai gépek, virtuális gépek vagy bármi más. Esetünkben ez egy 
virtuális gép lesz, melyet a Minikube nevezetű alkalmazás fog számunkra biztosítani. 
További információ a Minikube-ról. ~\cite{minikube}

Most, hogy a számunkra fontosabb részeket ismertettem áttérek a kubernetes 
hálózati modelljének bemutatására. Kezdetben három tulajdonságát szeretném 
bemutatni. 
\begin{itemize}
	\item Az összes pod képes kommunikálni a hálózatban megtalálható összes 
	pod-al NAT (Network Address Translation) használat nélkül 
	\item Az összes node képes kommunikálni az összes pod-al NAT nélkül 
	\item Amilyen IP címet lát a pod a saját interfészéhez rendelve, ugyan 
	azt a címet fogja látni más pod vagy node is a hálózatban. 
\end{itemize}
Így a következő hálózati kihívások jelentkeznek: 
\begin{enumerate}
	\item Konténer -> Konténer adattovábbítás 
	\item Pod -> Pod adattovábbítás 
	\item Pod -> Szolgáltatás adattovábbítás 
	\item Internet -> Szolgáltatás adattovábbítás 
\end{enumerate}
Ami most számunkra fontosabb lesz az az első kettő pont. 

Itt kicsit jobban fejtsd ki ez alapján: \\
\url{https://sookocheff.com/post/kubernetes/understanding-kubernetes-networking-model/} 

\subsubsection{Szolgáltatásháló}
A szolgáltatásháló, olyan mint mondjuk az Istio egy módja annak, hogy az 
alkalmazás különböző részei, hogyan osztanak meg adatot egymás között. 
Más rendszerekkel ellentétben a kommunikációra a szolgáltatásháló egy 
dedikált infrastruktúra réteg magában az alkalmazásban. Ez egy "látható" 
vagyis érzékeljük, hogy ott van, miközben a különböző részek kommunikálnak 
egymással és így láthatjuk, hogy ezek a komponensek hogyan működnek vagy 
nem és így egyszerűbbé válik a kommunikáció optimalizálása és elkerülhető 
a kiesett idő, amíg esetleg az alkalmazásunk nem üzemel. 

Minden részét az alkalmazásnak egy szolgáltatásnak hívunk, amelyek 
különböző folyamatokért felelősek. 

Legjobban úgy lehet megérteni, hogy hogyan működnek a szolgáltatáshálók, ha 
ezt egy ábrával tesszük. 
\begin{center}
\includegraphics[width=\textwidth]{serviceMesh}
\end{center}
Minden szolgáltatás egy mikroszolgáltatásból és egy sidecar proxy-ból áll. 
Mikroszolgáltatás alatt az alkalmazás egy adott funkciójának megvalósítását 
valamilyen nyelven értjük. Fontos kihangsúlyozni, hogy nem kötött milyen 
programozási nyelven van az adott mikroszolgáltatás írva, mert ez is 
nagyon rugalmassá teszi az ilyen fejlesztést, abban az esetben, ha 
különböző csapatoknak más és más nyelven akarnak megírni egyes szolgáltatásokat. 
Viszont itt jön képbe a sidecar proxy, ami elvégzi az a hálózati 
kommunikációt a szolgáltatások között. Szóval nem kell a fejlesztőnek 
a mikroszolgáltatásban azzal foglalkozni, hogy kiknek címezze a kimenő 
forgalmat vagy, hogy honnét kapja, mert ez mind a két esetben a sidecar lesz. 

A kép forrása és további érdekes információk megtalálhatóak még a Red Hat ezen 
cikkében ~\cite{redhat}

\subsubsection{Envoy}
Ez egy L7 (alkalmazás) rétegen működő proxy és kommunikációs csatorna 
arra tervezve, hogy nagy szolgáltatás alapú architektúrákat lehessen létrehozni 
úgy, hogy a szolgáltatások számára a hálózat átlátszó legyen.  
A proxy egy olyan szerver, amely hálózati forgalmat irányítja. Az előzőleg 
említett funkcióját úgy sikerül ellátnia, hogy a szolgáltatáshálók 
részben lévő szolgáltatásoknál az Envoy lesz maga a sidecar, így a 
mikroszolgáltatásnak elég mindig csak a localhost-ra küldeni és onnét 
fogadni adatot. 

Képes L3/L4 (hálózati/szállítási) réteg szűrőket is alkalmazni, amivel a 
különböző protokollokat, mint a TCP (Transmission Control Protocol) és az 
UDP. 

Ezen felül rendelkezik még olyan funkciókkal, amelyek segítségével 
megvalósítható perem proxy-ként és belső proxy-ként is, felügyelhető, 
és HTTP (HyperText Transfer Protocol) alapú szűrést is tudunk használni. 

További információk ezen az oldalon találhatóak ~\cite{envoydoc}

\subsubsection{További technológiák, amelyek szükségesek}
\textbf{UDP - User Datagram Protocol} \\
Az UDP egyszerű kapcsolatmentes összeköttetést hoz létre 2 eszköz 
között, így nincs kapcsolat ellenőrzés, mint a TCP-nél ezért sokkal 
gyorsabban felállítható a kapcsolat eszközök között. Viszont nincs 
garancia arra, hogy a csomagok megérkeznek, sorrendben érkeznek-e 
vagy, hogy duplikált csomagok jönnek át a csatornán. Ezért olyan 
szolgáltatásokhoz érdemes használni, ahol nem probléma, ha egy 
csomag többször jön vagy hiányoznak csomagok. Ilyenek például a 
telekommunikáció és az élő közvetítések is. További információk 
találhatóak ezen az oldalon ~\cite{udpwiki} \\

\textbf{UDS - Unix Domain Socket} \\ 
Arra használatos, hogy a folyamatok úgy tudjanak egymással kommunikálni, 
hogy az minél jobb legyen. A Unix Domain Socket lehet névtelen, vagy 
hozzárendelve egy fájlrendszerben létrehozott elérési útvonalhoz. 
További információkat lehet találni ezen az oldalon ~\cite{udsman} \\

\textbf{iPerf} \\ 
Az iPerf egy olyan eszköz, amellyel lehet mérni a maximálisan elérhető 
sávszéllességet IP hálózatokban. Támogatja a különböző protokollokat, így 
használható TCP és UDP mérésre is, de ezen felül lehet még sok máshoz is használni. 
De betudunk állítani olyanokat is, hogy adott ideig és időközökkel küldjön 
általunk megadott csomagmérettel és küldési sebességgel, szóval egy 
nagyon kis hasznos eszköz. 

Minden ilyen teszt végén kapunk egy jelentést arról, hogy milyen sávszélességgel 
érkeztek meg a csomagokat, ebből mennyi veszett el és még más paramétereket is 
annak tükrében, hogy milyen részletes jelentések szeretnénk látni. 

A használata úgy néz, hogy létre kell hozni egy úgynevezett iPerf szervert, 
amihez beérkeznek a csomagok. Ez lehet bárhol, amit az hálózaton keresztül elérünk. 
Ezen felül még létre kell hozni egy iPerf klienst is, aki az általunk 
specifikált módon fog csomagokat küldeni a szerverre.

További érdekes információk találhatóak az weboldalán ~\cite{iperf} \\

\textbf{socat} \\
A Socat egy parancssori alapú segédprogram, amely két kétirányú bájt-folyam 
hoz létre és adatot továbbít közöttük. Mivel az adatfolyamok különféle típusú 
adatgyűjtőkből és forrásokból állíthatóak össze, és mivel sok címbeállítási 
lehetőséget lehet alkalmazni az adatfolyamokra, a socat felhasználható 
többféle célra.

Ezt fogjuk arra használni, hogy UDS csomagokat alakítsunk át UDP csomagokká. 
További információk találhatóak a következő oldalon ~\cite{socat}
%Amikor a tágabb tudományos vagy műszaki környezetről beszélünk, akkor
%azt értjük alatta, hogy ,,művelt laikus'' -- alapesetben ennek
%tekinthető a tárgyfelelős -- megtalálja azokat a kapcsolódási
%pontokat, amelyek segítségével az ő ismereteihez a beszámolóban
%tárgyalt témakör és az elvégzett munka csatlakoztatható.  Ez
%mindenképpen, szükséges, hiszen enélkül az olvasónak előzetes
%tájékozódást kellene végeznie a szűkebb szakterületen, hogy az
%elvégzett munka jellegét, súlyát, nehézségét meg tudja ítélni.

%Természetesen a terjedelmi korlátok miatt nem lehet teljes mértékben
%bemutatni az adott szűkebb szakterületet, ezért meg kell adni az
%érdeklődő olvasónak a lehetőséget a további tájékozódásra.  Részben
%erre szolgálnak a hivatkozások\footnote{A másik nagyon fontos céljuk
%  az állítások alátámasztása.  A lábjegyzetek használatát egyébként
%  nem érdemes túlzásba vinni, mert állandóan megtörik az olvasás
%  folyamatát (lásd például \cite{esterhazy}).  Akkor kell használni,
%  amikor a lábjegyzetben közlendő információ érdekes lehet, de nem
%  tartozik közvetlenül a tárgyhoz.  Mindenképpen kerülendő az irodalmi
%  hivatkozások lábjegyzetben való megadása.}, lábjegyzetek.

%Nagyon fontos, hogy abban az esetben, amikor a hallgató féléves
%munkájának egy része vagy egésze az, hogy tanul, hogy maga is
%ismerkedik azzal a szakterülettel, amelyen dolgozni fog, akkor az
%tudatosan törekedjen arra, hogy az Elméleti összefoglaló és az
%Elvégzett munka ismertetése szétválasztható legyen.  Az előbbinek a
%szakterület alapvető, általános ismereteit kell tartalmaznia,
%amelyekről a ,,művelt laikusnak'' részleges ismeretei lehetnek.  Míg
%az elvégzett munka leírásába azoknak a specifikus ismereteknek a
%bemutatása kerülhet, amelyek a később elvégzett vagy elvégzendő
%feladatokat konkrétan megalapozzák.

%További tanulmányozásra ajánljuk Eco professzor művét~\cite{eco}.

%A hivatkozások kezelésénél fontos, hogy mindig a mondat részeként
%tekintsünk rá, és ha szükséges, akár többször is hivatkozzunk meg egy
%forrást, de az első előfordulásakor mindenképpen.  Ugyanez érvényes a
%rövidítésekre: minden rövidítést a legelső előforduláskor magyarázni
%kell, később viszont használhatók a rövidített formák is.  Például: a
%Tiger Tree Hash (TTH) a hashelés egy speciális formája.

\subsection{A munka állapota, készültségi foka a félév elején}
\label{sec:munka-allap-kesz}
A témalaboratórium alkalmával már foglalkoztam a szolgáltatáshálók 
megismerésével és használatával, ahol az Isio és az SMI (Service Mesh 
Interface) különböző aspektusait hasonlítottam össze. Ennek során 
mélyebb rálátást sikerült nyernem a Kubernetes használatára és arra, 
hogy egy jó dokumentáció milyen nagy segítséget nyújthat azoknak, 
akik ebben az iparágban dolgoznak. 
%Ebben a részben lehet megadni a korábbi félévekben elvégzett munkát
%is.  Így világosan elkülöníthető az aktuális félévtől.  Nem kell
%hosszasan írni, de ne legyen felsorolás sem.  Egy bekezdésben kellene
%itt leírni, hogy foglalkoztál-e már ezzel a témával, a tanszéken
%dolgozott-e már valaki rajta.  Írd le azt is, hogy mit kaptál kézhez
%hozzá segítségként.

\newpage
%==================================================================
\section{Az elvégzett munka és az eredmények ismertetése}
\label{sec:az-elvegzett-munka}


\subsection{A munkám ismertetése logikus fejezetekre tagoltan}
\label{sec:a-munkam-ismert}
<Én magam (nem a társam) a félév során következőket olvastam el /
programoztam / készítettem el / teszteltem / dokumentáltam / néztem át
/ tanultam meg, stb.  Tételes leírása és felsorolása mindannak, ami a
félév során történt, alátámasztandó azon állításom a
konzulens/tárgyfelelős felé, hogy összességében mindent beleértve
tényleg dolgoztam a TVSZ szerint kreditenként 30 órát, azaz a heti 2
kontakt órás tárgy esetében min. $2,5*30 = 75$ munkaórát, illetve a
heti 6 kontakt órás tárgy esetében min. $8*30 = 240$ munkaórát\dots>

Ebben a részben a hallgató az általa elvégzett munkát mutatja
be. Hangsúlyosan a saját munka bemutatása a cél, hiszen a hallgató
ezzel igazolja a témavezető és a tárgyfelelős irányába, hogy --
folyamatosan fejlődve és egyre több és jobb munkát végezve -- a
szakdolgozatát/diplomadolgozatát képes lesz megírni.  A beszámoló nem
munkanapló, nem arra vagyunk kíváncsiak, hogy mit mikor csinált a
hallgató és mennyi időt töltött vele, hanem egy eredmény-centrikus
beszámolót szeretnénk olvasni.  De itt is fontos tudni, hogy
megosztott feladat esetén ki-mit csinált, mekkora részt vállalt.

Az egész beszámoló elkészítésénél törekedni kell a magyar nyelv
szabályainak követésére és a műszaki dokumentáció/tudományos közlemény
írásával kapcsolatosan kialakult közmegegyezés szerinti formai
követelmények betartására.  (Tehát nem kell többes számként hivatkozni
saját magunkra, kerülni kell a furcsa megfogalmazást, passzív és egyéb
kifacsart mondatszerkezeteket.  Az egy szót határozatlan névelőként
történő használatakor ne írjuk ki számként.)


A beszámoló természetesen nem csak szöveget tartalmazhat, hanem
képleteket, táblázatokat, ábrákat és még sok minden mást.  Ezek
kapcsán az alábbi elvek irányadók:
\begin{itemize}
\item Az ábráknak, képeknek és táblázatoknak mindig van számuk és
  címük. (A cím nem ennyi: ,,1. ábra'', hanem azt írd le, ami látható
  rajta.)

\item Az ábrákra, a képekre és a táblázatokra a szövegben hivatkozni
  kell, és a szövegben elemezni kell azokat. Például
  \aref{fig:fig1}.~ábrán látszik, hogy a vizsgált félévben még két
  napos csúszással is lehetett jeles érdemjegyet szerezni a tárgyból,
  de a pontosság még nem garancia a jó jegyre: öten nem kaptak jelest,
  noha nem késtek a leadással.

\item Az ábrák, képek és táblázatok mérete a szükségesnek megfelelő
  legyen: elég nagy ahhoz, hogy kinyomtatva is olvasható és
  értelmezhető legyen, de nem nagyobb annál, mint amit szerepe
  indokol.

\item A grafikonoknak a tengelyeken legyenek feliratai és ha releváns,
  a mértékegység is.

\item A képletek esetében nem minden képletre történik hivatkozás, de
  ahol igen, ott a képletet a műszaki irodalomban jellemző módon a sor
  végére tett kerek zárójelben lévő számmal jelöljük meg.  A
  képleteket ne képként illeszd be a szövegbe.

\item Kódrészleteket, ha nem relevánsak, ne illeszd be képként, főleg
  ne rossz minőségben. Nyugodtan teheted függelékbe és hivatkozd be a
  szövegben, mint a képeket, például: Az 1.~számú függelékben
  található az adatbeolvasó kód, melyet C++ nyelven készítettem el.
\end{itemize}

Az írásbeli beszámolót a témavezető és a tárgyfelelős is értékeli. A
tárgyfelelősi értékelés szempontjai az alábbiak:
\begin{enumerate}
\item Megfelel-e az elvégzett munka a félév elején kiadott feladatnak?
\item Megfele-e a beszámoló a formai követelményeknek? Ezen belül:
  \begin{itemize}
  \item a. Megfelelő-e az elméleti bevezető és az irodalomjegyzék?
  \item b. Egyértelmű-e, hogy mi volt a hallgató saját munkája?
  \item c. Megfelelő-e a dokumentum technikai színvonala?
  \end{itemize}

\end{enumerate}
Ezen kívül a tárgyfelelős veszi figyelembe az értékelés során
kialakult félévi jegyre vonatkoztatva az ún. ,,hanyagsági faktor''
értékét, amelyet (\ref{eq:1}) szerint állapítunk meg:

\begin{equation}
    F_{hany} = 1 - a - b
  \label{eq:1}
\end{equation}


\begin{figure}[tbh]
  \centering
  \includegraphics[]{fig1.eps}
  \caption{Hallgatók érdemjegyeinek eloszlása az írásbeli beszámoló késése függvényében}
  \label{fig:fig1}
\end{figure}

Az (\ref{eq:1})-ben szereplő a szám a munkaterv beadásában történt
késedelemre, míg a b szám az írásbeli beszámoló beadásában történt
késedelemre vonatkozik.  Utóbbi értékeiről
\aref{tab:hanyagsagi}.~táblázat tájékoztat.

\begin{table}
  \centering
    \begin{tabular}{|l|c|}
      \hline
      Az írásbeli beszámoló beadásának napja     & A ,,b'' faktor értéke \\
      a szóbeli beszámolóhoz képest (munkanapban) & ~ \\ \hline          
      -4.~munkanap & $0.04$ \\ \hline 
      -3.~munkanap & $0.09$ \\ \hline 
      -2.~munkanap & $0.20$ \\ \hline 
      -1.~munkanap & $0.30$ \\ \hline
      \end{tabular}
    \caption{Az írásbeli beszámoló késedelmes beadásával kapcsolatos hanyagsági faktor értéke}
  \label{tab:hanyagsagi}
\end{table}

A beszámoló értékeléséről részletesebben írunk \cite{web}-ban.

A beszámolóban bizonyára szerepelni fognak rövidítések. Ezeket a
rövidítéseket, betűszavakat néhány, az infokommunikáció területén
nagyon ismert és gyakran használt kifejezéstől (például IP, TCP, GPRS,
UMTS) eltekintve ki kell fejteni logikusan az első használat
alkalmával (például így: ,,A GPS (Generalized Processor Sharing) egy
ideális folyadékmodellen alapuló csomagütemező eljárás.'').

A beszámoló készítése során előfordulhat, hogy a hallgató úgy érzi,
hogy alfejezetekkel tagolva jobban olvasható és érthető lenne a
beszámoló.  Ennek akadálya nincs, de érdemes arra figyelni, hogy a
túlzott tagolás sem tesz jót egy írásműnek, illetve hogy a címsorokban
a rövidítések és a hivatkozások használata tilos.  Tartalomjegyzéket
készíteni nem szükséges a beszámolóhoz, de nem is tilos, kivéve azt az
esete, amikor nyilvánvalóan terjedelemnövelési célokat szolgál.

A beszámoló terjedelme tárgyanként változhat.  Általános szabály, hogy
1 hüvelyknél nagyobb margókat ne használjunk.  A szöveg legyen
egyszeres sortávú, sorkizárt és 12 pontos betűméretű.  A bekezdések
kezdődjenek behúzással a minta szerint.

\subsection{Összefoglalás}
\label{sec:osszefoglalas}

A félévi munka során elért új eredmények ismételt, vázlatos, tömör
Ebben a részben az adott félévre vonatkozó, az \emph{Önálló
  laboratórium tárgy keretében elvégzett munka során} elért
\textbf{új} eredmények ismételt, vázlatos, \textbf{tömör}
összefoglalását várjuk, lehetőleg nem felsorolásként.  Itt még egyszer
ki lehet térni a leglényegesebb eredményekre, valamint a félév során
felmerülő nehézségekre, de meg lehet említeni a továbbfejlesztési
irányokat, lehetőségeket is.

Ezt a részt tagolható a következő pontok megválaszolásával:
\begin{itemize}
\item Mi volt az \textbf{aktuális kérdés}, probléma, amivel a félév
  során foglalkoztál?
\item Mi a dolgozat \textbf{célja}, miért érdekes egyáltalán ezzel a
  problémával foglalkozni?
\item Milyen \textbf{módszereket} használtál a probléma megoldása
  érdekében?
\item Mik a legfontosabb \textbf{eredmények}?
\item Milyen \textbf{következtetéseket} lehet levonni?

\end{itemize}

Ha valaki elolvassa ezt a részt, képet kell kapnia az egész
dolgozatról.  Ne legyen az absztrakt szó szerinti ismétlése.

Fontos, hogy az itt megadott sablontól el lehet térni, használata nem
kötelező, csak segítséget jelenthet, viszont a fedőlap lehetőleg
maradjon ugyanez és tartalmilag egyezzen meg a sablon irányelveivel. A
beszámoló felépítésében nem érdemes eltérni a \emph{Bevezető --
  Féléves munka és eredmények bemutatása -- Összefoglaló} hármastól.

\newpage
 
%==================================================================
\section{Irodalom, és csatlakozó dokumentumok jegyzéke}
\label{sec:irod-es-csatl}

\begin{thebibliography}{9}
\label{sec:tanulm-irod-jegyz}

\bibitem{linuxwindows} Microsoft Corporation, \emph{Linux containers on Windows 10} \\
\url{https://docs.microsoft.com/en-us/virtualization/windowscontainers/deploy-containers/linux-containers}

\bibitem{dockerhub}  Docker Inc, \emph{DockerHub} \url{https://hub.docker.com/}

\bibitem{dockeroff}  Docker Inc, \emph{Docker Documentation} \url{https://docs.docker.com/}

\bibitem{dockerwiki} Wikipedia contributors, \emph{Wikipedia:Academic
    use}, Wikipedia, The Free Encyclopedia, 2011 Nov 11.  Available
  from: \\ \url{https://en.wikipedia.org/wiki/Docker_(software)}

\bibitem{minikube} The Kubernetes Authors, \emph{Minikube documentation} \url{https://minikube.sigs.k8s.io/docs/} 

\bibitem{redhat} Red Hat Inc, \emph{What's a service mesh?} 
\\ \url{https://www.redhat.com/en/topics/microservices/what-is-a-service-mesh}

\bibitem{envoydoc} Envoy Project Authors, \emph{What is Envoy} 
\\ \url{https://www.envoyproxy.io/docs/envoy/latest/intro/what_is_envoy}

\bibitem{udpwiki} Wikipedia contributors, \emph{User Datagram Protocol}
\\ \url{https://en.wikipedia.org/wiki/User_Datagram_Protocol}

\bibitem{udsman}  Linux man-pages project, \emph{Unix Domain Socket}
\\ \url{http://man7.org/linux/man-pages/man7/unix.7.html}

\bibitem{iperf}  iPerf contributors, \emph{iPerf} \url{https://iperf.fr/}

\bibitem{socat}  Gerhard Rieger, \emph{socat - Linux man page} \url{https://linux.die.net/man/1/socat}

\bibitem{esterhazy} Esterházy Péter, \emph{Termelési-regény (Kisssregény),}
  Magvető Könyvkiadó, 2004, ISBN: 9631423948.

\bibitem{web} \emph{Tájékoztató a Műszaki Informatika Szak önálló
    laboratórium tantárgyainak 2008/9. tanév I. félévi lezárásáról a
    BME TMIT-en (VITMA367, VITMA380, VITT4353, VITT4330),}
  \url{http://inflab.tmit.bme.hu/08o/lezar.shtml}, szerk.: Németh Felicián,
  2008. november 5.

\bibitem{wikipedia} Wikipedia contributors, \emph{Wikipedia:Academic
    use}, Wikipedia, The Free Encyclopedia, 2011 Nov 11.  Available
  from: \\ \url{http://en.wikipedia.org/w/index.php?title=Wikipedia:Academic\_use\&oldid=460041928}

\end{thebibliography}

Itt jegyezném meg, hogy a tanulmányozott irodalmat hivatkozni kell a
szövegben.  Szükség esetén többször is.  Az irodalomjegyzék célja
(lásd \aref{sec:tanulm-irod-jegyz} fejezetet) ugyanis
kettős\footnote{Akárcsak ennek a fejezet hivatkozásnak, ami a
  \texttt{$\backslash$aref babel} parancsot demonstrálja}:
\begin{enumerate}
\item Az olvasó tájékoztatása, hogy a dokumentumban ki nem fejtett
  dolgoknak, a tudottnak vélt ismereteknek hol lehet bővebben
  utánanézni, így ott kell meghivatkozni az irodalmat~\cite{eco,
    esterhazy}, ahová az irodalom kapcsolódik.
\item Megmutatni a tárgyfelelosnek/konzulesnek az elolvasott irodalom
  mennyiségét
\end{enumerate}

Javasoljuk, hogy a hallgatók tanulmányozzák, hogyan néznek ki a
hivatkozások a villamosmérnöki/informatikai szakma vezető szakmai
folyóirataiban megjelenő cikkekben.  Ebben a témavezető is biztosan
tud segíteni.  A hivatkozás teljességére és egyértelműségére tessék
ügyelni.  Például, ha egy könyvnek több, eltérő kiadása is van, akkor
azt is meg kell jelölni, hogy melyik kiadásra hivatkozunk.  A webes
hivatkozások problémásak szoktak lenni, de manapság egyre több az
olyan dokumentum, ami csak weben lelhető fel, ezért használatuk nem
zárható ki. Itt is törekedni kell azonban a pontosságra és a
visszakereshetőségre. A weben található dokumentumoknak is van címe,
szerzője, illetve érdemes megadni a letöltés/olvasás időpontját is,
hiszen ezek a dokumentumok idővel megváltozhatnak.

A wikipédiás hivatkozások használata nem javasolt, mert a wikipedia
másodlagos forrás.  Tájékozodjuk a wikipédián, de aztán olvassuk el az
adott oldalhoz megadott hivatkozásokat is.  A wikipedián külön szócikk
foglalkozik azzal, hogy miért nem szerencsés tudományos munkákban a
wikipédiára hivatkozni \cite{wikipedia}.

Nem publikus dokumentumok hivatkozása nem javasolt és csak kivételes
helyzetben elfogadható!

%==================================================================
\subsection{A csatlakozó dokumentumok jegyzéke}
\label{sec:csat-irod}

<A munka ezen beszámolóba be nem fért eredményeinek (például a forrás
fájlok, mindenképpen csatolni akart forráskód részlet, felhasználói
leírások, programozói leírások (API), stb.) megnevezése,
fellelhetőségi helyének pontos definíciója, mely alapján a az
erőforrás előkereshető -- értelemszerűen nem nyilvános dokumentumok
hivatkozása nem elfogadható.>

\end{document} 

%%% Local Variables: 
%%% mode: latex 
%%% TeX-master: t 
%%% End:

